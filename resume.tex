\documentclass{article}
\usepackage{enumerate, amssymb}
\usepackage[margin=.7in]{geometry}
\usepackage{scrextend}
\usepackage{indentfirst}
\usepackage{fancyhdr}
\usepackage[T1]{fontenc}
\usepackage[utf8]{inputenc}

\newcommand{\code}[1]{\texttt{#1}} % Monospace font (code etc.)

% NOTE: Commented out sections may either be considered too old or there is simply not enough space

\thispagestyle{empty}
\begin{document}
	\begin{center}
	
		\huge{Brady Bolton}
		
		\large{bradybw@vt.edu}

		\large{\code{www.github.com/BradyBolton}}

        % TODO: maybe make a footer mentioning that the source can be found in github
        %       since there's a lot of stuff commented out to make this resume 1 page

	\end{center}

\vline


\textbf{\large{Education}} \quad Computer Science BSc at Virginia Tech

\qquad \qquad \qquad \hspace{10pt} Computational Modelling \& Data Analytics BSc (Secondary Degree)

\qquad \qquad \qquad \hspace{10pt} Mathematics and Statistics Minors

\qquad \qquad \qquad \hspace{10pt} 3.64 GPA (\textit{Magna Cum Laude})

\vline

\textbf{\large{Languages}} \quad Golang, TypeScript/JS, C\#, Python, SQL, Java, C, R, MATLAB, \LaTeX, Lisp

\textbf{\large{Skills}} \qquad \qquad Linux, React, .NET MVC, Django/DRF, Bash/sh, SASS, Git, Docker/Docker-Compose, K8s,

\qquad \qquad \qquad \hspace{10pt}  AWS, ADO, gRPC/proto3, Postgres, SQLServer, Snowflake, CUDA/MPI/MPICC,  BACnet %(IP/MSTP)

\vline

\textbf{\large{Related Experience}} 

\vspace{5pt}

{\setlength{\leftskip}{15pt}

{\fontfamily{qcs}\selectfont Associate Developer, Koalafi, Spring 2022-Present}
\vspace{-5pt}
\begin{itemize}
	\setlength{\leftskip}{15pt}
	\setlength\itemsep{-0.5em}
    \item[$-$] Late-stage fintech startup offering frictionless financing solutions to brick-and-mortar \& ecommerce businesses
    \item[$-$] Feature ownership for React front-ends, Golang microservices, and C\# .NET MVC monoliths, from IDE to deployment (AWS, ADO, K8s), monitoring (ELK \& TICK stacks, Dynatrace, Segment.io), and prod support
    \item[$-$] Serverless development with AWS Lambda, integrating with OpenTelemetry, SQS/SNS, and Slack to drive time-sensitive business operations via event-based architecture
\end{itemize}

{\fontfamily{qcs}\selectfont Software Engineering Intern, NetApp, Summer 2021}
\vspace{-5pt}
\begin{itemize}
	\setlength{\leftskip}{15pt}
	\setlength\itemsep{-0.5em}
	\item[$-$] Wrote a containerized full-stack web-app for discovering and managing switch firmware
	\item[$-$] Requests routed through Nginx reverse-proxy and Gunicorn WSGI deployed with docker-compose
	\item[$-$] Implemented Django + Django-Rest-Framework back-end to handle user requests in parallel using asyncio
\end{itemize}


{\fontfamily{qcs}\selectfont Software Engineering Intern, Progeny Systems, Summer 2020}
\vspace{-5pt}
\begin{itemize}
	\setlength{\leftskip}{15pt}
	\setlength\itemsep{-0.5em}
	\item[$-$] Worked with microservices in a cloud-native stack implemented in Golang, proto3, and gRPC 
	\item[$-$] Troubleshooted container networks with \code{tcpdump}, \code{smcroute}, and \code{iptables}
	\item[$-$] Migrated a docker-compose setup to Kubernetes using Helm
\end{itemize}


{\fontfamily{qcs}\selectfont Software Engineering Co-op, Daikin Applied Americas, Spring 2020 – Summer 2020}
\vspace{-5pt}
\begin{itemize}
	\setlength{\leftskip}{15pt}
	\setlength\itemsep{-0.5em}
	\item[$-$] Building automation systems (BAS) integration testing using BACnet IP/MSTP, MODBUS, Webkit 
	\item[$-$] Used Wireshark and PCAP to investigate networking bugs on BAS systems
	\item[$-$] Focused on Python best-practices, development, and documentation using PyTest, Cython, and Sphinx
\end{itemize}

% {\fontfamily{qcs}\selectfont Controls Programmer, Virginia Tech Hyperloop, Fall 2019}
% \vspace{-5pt}
% \begin{itemize}
% 	\setlength{\leftskip}{15pt}
% 	\setlength\itemsep{-0.5em}
% 	\item[$-$] Using QT platform facilitating real-time sensor data transfer to a remote GUI and communicate with SpaceX's track telemetry system via UDP
% 	\item[$-$] Working closely with aerospace and mechanical sub-teams to understand and simulate pod mechanics
% \end{itemize}

{\fontfamily{qcs}\selectfont Undergraduate Research Assistant, Socha Labs, Spring 2019 – Summer 2019}
\vspace{-5pt}
\begin{itemize}
	\setlength{\leftskip}{15pt}
	\setlength\itemsep{-0.5em}
	\item[$-$] Responsible for programming and testing servo motor controllers and data-acquisition for robotic fish
	\item[$-$] Worked closely with faculty at the Socha Lab of Biomedical Engineering and Mechanics (BEAM) to empirically study bio-mechanical models of fish locomotion and its effect on speed and mechanical efficiency
\end{itemize}

% {\fontfamily{qcs}\selectfont Co-founder and Lead Designer, Zorse Code LLC, Summer of 2017 – Summer 2018}
% \vspace{-5pt}
% \begin{itemize}
% 	\setlength{\leftskip}{15pt}
% 	\setlength\itemsep{-0.5em}
% 	\item[$-$] Co-founded a gaming start-up that incorporates augmented reality 
% 	\item[$-$] Designed and produced Graphical User Interface, artwork, and navigation system
% \end{itemize}

% \vline

% \textbf{\large{Interests}} 
% 
% \vspace{3pt}
% 
% \begin{tabular}{l l}
% 	Networking and Self-Hosting: running NextCloud, WireGaurd, PiHole on a small home-lab \\
% 	Linux: Debian as daily-driver, ZFS, and an undisclosed amount of dot-file tweaking \\
% 	Other tech-related topics: Emacs/SpaceMacs, Org-mode, NeoVim \\
% \end{tabular} 
% 
% \vline
% \vspace{0.5cm}

% \textbf{\large{Ongoing Projects}} 
% 
% \vspace{5pt}
% 
% {\setlength{\leftskip}{15pt}
% 
%       % MicroPython implementation on ESP8266 via Micropython
% 	{\fontfamily{qcs}\selectfont My Dog Sends Me Texts}
% 	\vspace{-5pt}
% 	\begin{itemize}
% 		\setlength{\leftskip}{15pt}
% 		\setlength\itemsep{-0.5em}
% 		\item[$-$] Make-shift capacitive touch pads that my dog presses to send SMS messages to family group-chat
% 		\item[$-$] Bestowed upon a small brown canine the power of a three word vocabulary: food, water, walk
% 		\item[$-$] Runs on an ESP8266 with callbacks tied to Twilio's API (using MicroPython)
% 	\end{itemize}
% 
% 	{\fontfamily{qcs}\selectfont Designated Driver App}
% 	\vspace{-5pt}
% 	\begin{itemize}
% 		\setlength{\leftskip}{15pt}
% 		\setlength\itemsep{-0.5em}
% 		\item[$-$] Front-end implemented with Next.js/React.js
% 		\item[$-$] Containerized Golang Fiber backend (Express.js inspired)
% 		\item[$-$] PWA implementation to support multiple platforms
% 	\end{itemize}
% }
% \setlength{\leftskip}{0cm}

\setlength{\leftskip}{0cm}
\textbf{\large{Projects}} 

\vspace{5pt}

{\setlength{\leftskip}{15pt}

	{\fontfamily{qcs}\selectfont RideChariot.app}
	\vspace{-5pt}
	\begin{itemize}
		\setlength{\leftskip}{15pt}
		\setlength\itemsep{-0.5em}
		\item[$-$] Uber-like app (PWA) to make ride-sharing quick and easy for Virginia Tech's fraternities
		% \item[$-$] Partnered with Alpha Sigma Phi for collaboration, user feedback, and beta-testing
		\item[$-$] Front-end in React with static site generation using Next.JS (targeting a Jamstack-inspired architecture)
		\item[$-$] Dockerized back-end with a \code{gofiber} internal API using MongoDB, Amazon SNS, and Traefik load-balancing
	\end{itemize}

        % MicroPython implementation on ESP8266 via Micropython
	{\fontfamily{qcs}\selectfont My Dog Sends Me Texts}
	\vspace{-5pt}
	\begin{itemize}
		\setlength{\leftskip}{15pt}
		\setlength\itemsep{-0.5em}
		\item[$-$] Make-shift capacitive touch pads that my dog presses to send SMS messages to family group-chat
		\item[$-$] Runs on an ESP8266 MCU calling Twilio's API from a local WiFi network (via MicroPython)
		\item[$-$] Bestow animals a rudimentary vocabulary (e.g. food, treat, water, walk)
	\end{itemize}

	% {\fontfamily{qcs}\selectfont CLI Cheatsheet for Aliases}
	% \vspace{-5pt}
	% \begin{itemize}
	% 	\setlength{\leftskip}{15pt}
	% 	\setlength\itemsep{-0.5em}
	% 	\item[$-$] A Golang ncurses-like tool to parse aliases and spit out an organized cheatsheet on the CLI
	% 	\item[$-$] Organize your aliases into categories by including directives in your dot files
	% 	\item[$-$] Supports configuration, paging, vim-inspired controls, and on-the-fly reformatting
	% \end{itemize}

	{\fontfamily{qcs}\selectfont Ray Tracing with Parallel Computing}
	\vspace{-5pt}
	\begin{itemize}
		\setlength{\leftskip}{15pt}
		\setlength\itemsep{-0.5em}
	\item[$-$] Rendered scenes on VT's compute clusters by simulating light rays via parallel processing
	\item[$-$] Implemented with MPI (Message Passing Interface), MPICH, and via Nvidia's CUDA platform using GPUs
	\end{itemize}
}
\setlength{\leftskip}{0cm}


\textbf{\large{Awards and Activities}}

\vspace{3pt}

\begin{tabular}{l l}
	ACM Programming Team % (Fall 2018)
    & ICPC Team Honorable Mention Award (2018) \\
	Hyperloop at Virginia Tech 	& Dean's List \\
	Philosophy Club 			& Virtual Entities (AI Programming) \\
	% 4th International MIT Zero Robotics Alliance	& Microsoft Imagine Cup Competition \\
\end{tabular} 

\end{document}
